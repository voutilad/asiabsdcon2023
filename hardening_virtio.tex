\documentclass[conference]{IEEEtran}
%\IEEEoverridecommandlockouts
% The preceding line is only needed to identify funding in the first footnote. If that is unneeded, please comment it out.
\usepackage{cite}
\usepackage{amsmath,amssymb,amsfonts}
\usepackage{algorithmic}
\usepackage{graphicx}
\usepackage{textcomp}
\usepackage{xcolor}
\usepackage{hyperref}
\usepackage{theorem}
\def\BibTeX{{\rm B\kern-.05em{\sc i\kern-.025em b}\kern-.08em
    T\kern-.1667em\lower.7ex\hbox{E}\kern-.125emX}}
\newtheorem{trait}{Ideal Trait}
\begin{document}

\title{Hardening Emulated Devices \\ in OpenBSD's vmd(8) Hypervisor}

\author{\IEEEauthorblockN{Dave Voutila}
\textit{OpenBSD}\\
dv@openbsd.org}

\maketitle

\begin{abstract}
The 2010s brought commoditization of hardware-assisted virtualization
as now most consumer operating systems and computers ship with both
support in hardware as well as Type-2 hypervisors. With hypervisors
comes the need for emulated devices to provide virtual machines
interfaces to the outside world, including network cards, disk
controllers, and even hardware random number generators. However,
hypervisors are still software programs and consequently subject to
buffer overflow and stack smashing attacks like any other. Previous
research has shown a common weak point in hypervisors to be these very
emulated devices where exploits enable ``guest to host escapes'', the
most famous being an exploit of an emulated floppy disk controller.

This paper provides an experimental approach with OpenBSD's Type-2
hypervisor (vmd) for isolating emulated devices using the priviledge
dropping and seperation capabilities available in OpenBSD 7.3 to
mitigate techniques for turning memory bugs into guest-to-host
hypervisor escapes.
\end{abstract}


\begin{IEEEkeywords}
OpenBSD, virtualization, security
\end{IEEEkeywords}

\section{Introduction}
One of the earliest documented guest-to-host escapes was made possible
by a buffer overflow of an emulated video card in the Xen hypervisor
(CVE-2008-1943 \cite{b1}). Since then, every major hypervisor whether
open-source or commercial, has had something in common: a buffer
overflow or unitialized memory bug in an emulated device allowing for
exploitation by a malicious guest operating system. This raises two
concerns:

\begin{itemize}
\item the attractiveness of attacking emulated devices in a hypervisor,
\item the accessibility of modern techniques to exploit these memory
  bugs when found.
\end{itemize}

\subsection{Emulated Devices as Targets in Hypervisors}
Like with real computers, devices form the interface between a virtual
machine and the outside world. While hardware-assisted virtualization
allow CPU-intensive tasks to achieve bare-metal speeds, at some point
the guest will need to perform I/O whether sending a network packet or
writing a block to persistent storage. Given I/O becomes a noticeable
performance bottleneck in virtualized systems, hypervisor authors
often optimize in multiple ways:

\begin{itemize}
\item emulate a device in the host kernel to reduce overhead
\item emulate all devices in the same process via threads
\item pass-through access to a real, physical device
\end{itemize}

As a result, emulated devices often require or evolve to acquire
elevated permissions and capabilities, making them a high-value target
for an attacker.

The high-valued nature isn't the only problem: emulated devices must
work with guest device \emph{drivers}. Hypervisor authors must create
virtual hardware from software and, while there exist
virtualization-specific specifications like VirtIO \cite{b2}, the
nature of I/O requires moving guest-supplied data back and forth. A
memory bug easily becomes ``remotely'' triggerable by a guest device
driver or a even a packet coming in to the host from outside.

Given the above, it should come as little surprise that almost all
published virtual machine ``escapes'' \cite{b3} have been as the
result of exploiting emulated devices!


\subsection{Return-Oriented Programming}
At the same time researchers began to find the first guest-to-host
exploits in hypervisors, other researchers found novel ways to go
beyond simple code-injection techniques and use a program's code
against itself.

Return-Oriented Programming (ROP) \cite{b3} provides a manner to
defeat W$\oplus$X mitigations, commonplace in operating systems with
the support of a hardware ``no-execute'' (NX) page protection bit. ROP
attacks use either the program itself or runtime libraries like
\emph{libc} to execute arbitrary code assembled by finding and
leveraging specificly useful machine instructions called ``gadgets.''
These gadgets, when executed in a particular order (called a
``chain''), allow an attacker to achieve arbitrary code execution.

Multiple approaches exist to help prevent a successful ROP attack,
including Adress Space Randomization (ASLR) and Control Flow Integrity
(CFI) \cite{b4}.

\subsection{Blind Return-Oriented Programming}
In 2014, \emph{Bittau et. al.} showcased an evolution of ROP, called
``Blind Return-Oriented Programming'' (BROP) \cite{b5}, designed to
overcome ASLR techniques and remotely exploit stack buffer overflows
through information leakage. The authors' techniques of ``stack
reading'' to defeat ASLR and remotely generate ROP chains (sequences
of gadgets) to achieve arbitrary code execution emphasized the
severity of stack or heap overflows in programs: \emph{if it's
remotely accessible and doesn't re-randomize itself on a restart, any
remotely triggerable stack buffer overflow provides a BROP attack
vector.}

How do these concerns apply to hypervisors? While there are no
currently known successful BROP-based attacks on hypervisors, the
concept of \emph{information leakage} still applies. If an attacker
can use a vulnerable emulated device to leak the hypervisor code into
guest memory (assuming no program crash), it's possible to perform a
ROP analysis.


\section{Defensive Concerns of the Ideal Hypervisor}
Considering the state-of-the-art of ROP-based attacks, what would the
``ideal'' hypervisor consider? BROP has shown it's important to
consider the following:

\begin{enumerate}
\item Information leakage allows for defeating ASLR defenses.
\item ROP gadgets allow assembling any needed system call.
\item System calls allow lateral movement to take over a host.
\end{enumerate}

The ideal hypervisor would maximize the complexity of defeating
ASLR. While the trivial approach would be to simply not respawn after
a crash (which isn't a common hypervisor trait anways), one can
propose:

\begin{trait}
An information leak in one vulnerable component of a hypervisor must
not inform on other components of the hypervisor.
\end{trait}

And what about ROP gadgets? They're impossible to completely remove
from the x86 architecture, so while they can be minimized by changes
to compilers, in the ideal case our hypervisor would minimize their
value:

\begin{trait}
Compromising a component of a hypervisor must not allow for
compromising other components of the hypervisor, i.e. a vulnerable
network device must not allow for compromising other devices.
\end{trait}

Lastly, keeping with the principle of least privledge:

\begin{trait}
Escaping a guest, via any means, must force the attacker to now
exploit the hypervisor itself to gain control of the host.
\end{trait}

While we can harden emulated devices, the difficulty posed to the
attacker should stay constant or increase even if they manage to take
control of the device.


\section{OpenBSD's Hypervisor}
Available since OpenBSD 5.9, released on 29 March, 2016, OpenBSD's
hypervisor consists of three parts:

\begin{itemize}
\item \texttt{vmm(4)} - the in-kernel virtual machine monitor
\item \texttt{vmd(8)} - the userland virtual machine daemon
\item \texttt{vmctl(8)} - a utility for interacting with vmd
\end{itemize}

This paper focuses primarily on \texttt{vmd(8)} because it provides
the emulated devices of interest (e.g. VirtIO network devices).

\subsection{Priviledge Separation in \texttt{vmd}}
While not in the original release of \texttt{vmd}, Reyk Floeter
committed \cite{b6} a redesign to implement a \emph{fork+exec} model
like the one already used in other OpenBSD daemons such as
\texttt{httpd(8)}.

...

\subsection{Priviledge Dropping in \texttt{vmd}}
As of OpenBSD 7.2, \texttt{vmd} incorporates three (3) primary
mechanisms for dropping priviledge:

\begin{enumerate}
\item \texttt{setresuid(2)/setresgid(2)} - for changing uid/gid
\item \texttt{chroot(2)} - for isolating file system access
\item \texttt{pledge(2)} - for removing system call access
\end{enumerate}

...

\subsection{Existing Weaknesses in \texttt{vmd}}
Even with the existing PrivSep design and mitigations, the following
remains true for \texttt{vmd} as of OpenBSD 7.2:

\begin{enumerate}
\item All devices are emulated in the same guest vm process.
\item Guest vm process creation relies only on \texttt{fork(2)}, meaning address layouts are the same across guest vms.
\item \texttt{vmd} was designed to allow granting users management of specific vm's, including stopping and starting.
\end{enumerate}

Since BROP only requires a remotely exploitable stack overflow (possible via a bug in our emulated device) and the \emph{lack of re-randomization of the address space between crashes}, it's feasible to say with the proper configuration, \texttt{vmd} is open to a local BROP-based attack by a user granted control over a single vm.

...

\section{Hardening a \texttt{vmd} Device}
So what can we change to mitigate BROP? Removing the ability for users to start or stop vm's, thus removing the ``restart on a crash'' requirement for BROP, would reduce the utility of \texttt{vmd}. Instead, we'll focus on the other half of the same requirement: re-randomization.

We'll focus on weakenesses (2) and (1) in that order.

\subsection{Maximizing Randomness}
To make each guest have their own address space layout, it should be
as simple as performing the \texttt{exec(2)} part of ``fork+exec,''
right? \texttt{vmd} poses some challenges towards implementation \emph{because of its existing PrivSep design!}

...

\subsection{Minimizing the Impact}
\texttt{vmd} has had its share of security errata and, like most
hypervisors, most have been related to emulating a network device. How
can we minimize the impact and make BROP more difficult?

...

\section{Future Work and Considerations}

...

\section{Conclusions}

...

\section*{Acknowledgment}

TBA

\begin{thebibliography}{00}
\bibitem{b1} Red Hat Bugzilla, \url{https://bugzilla.redhat.com/show_bug.cgi?id=443078}, accessed December 2023.
\bibitem{b2} Virtual I/O Device (VIRTIO) Version 1.1. Edited by Michael S. Tsirkin and Cornelia Huck, 11 April 2019, OASIS Committee Specification 01, \url{https://docs.oasis-open.org/virtio/virtio/v1.1/cs01/virtio-v1.1-cs01.html}.
\bibitem{b3} Wikipedia, ``Virtual machine escape'', \url{https://en.wikipedia.org/wiki/Virtual_machine_escape#Previous_known_vulnerabilities}, retrieved 26 December, 2022.
  \bibitem{b4} V. Pappas, ``Defending against Return-Oriented Programming'', Columbia University, 2015.
\bibitem{b5} H. Shacham, ``The Geometry of Innocent Flesh on the Bone:
  Return-into-libc without Function Calls (on the x86)'', Proceedings of the CCS 2007, ACM Press, pp. 552-61, 2007.
\bibitem{b6} OpenBSD GitHub mirror, \url{https://github.com/openbsd/src/commit/bcc679a146056243a2fd52a28182621f893fed4b}
\end{thebibliography}

\end{document}
