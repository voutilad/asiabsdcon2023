\documentclass[conference]{IEEEtran}
%\IEEEoverridecommandlockouts
% The preceding line is only needed to identify funding in the first footnote. If that is unneeded, please comment it out.
\usepackage{cite}
\usepackage{amsmath,amssymb,amsfonts}
\usepackage{algorithmic}
\usepackage{graphicx}
\usepackage{textcomp}
\usepackage{xcolor}
\usepackage{hyperref}
\usepackage{theorem}
\def\BibTeX{{\rm B\kern-.05em{\sc i\kern-.025em b}\kern-.08em
    T\kern-.1667em\lower.7ex\hbox{E}\kern-.125emX}}
\newtheorem{trait}{Ideal Trait}
\begin{document}

\title{Hardening Emulated Devices \\ in OpenBSD's vmd(8) Hypervisor}

\author{\IEEEauthorblockN{Dave Voutila}
\textit{OpenBSD}\\
dv@openbsd.org}

\maketitle

\begin{abstract}
The 2010s brought commoditization of hardware-assisted virtualization
as now most consumer operating systems and computers ship with both
support in hardware as well as Type-2 hypervisors. With hypervisors
comes the need for emulated devices to provide virtual machines
interfaces to the outside world, including network cards, disk
controllers, and even hardware random number generators. However,
hypervisors are still software programs and consequently subject to
buffer overflow and stack smashing attacks like any other. Previous
research has shown a common weak point in hypervisors to be these very
emulated devices where exploits enable ``guest to host escapes'', the
most famous being an exploit of an emulated floppy disk controller.

This paper provides an experimental approach with OpenBSD's Type-2
hypervisor (vmd) for isolating emulated devices using the priviledge
dropping and separation capabilities available in OpenBSD 7.2 to
mitigate techniques for turning memory bugs into guest-to-host
hypervisor escapes.
\end{abstract}

\vspace{2mm}
\begin{IEEEkeywords}
OpenBSD, virtualization, security
\end{IEEEkeywords}

\vspace{5mm}
\section{Introduction}
One of the earliest documented guest-to-host escapes was made possible
by a buffer overflow of an emulated video card in the Xen hypervisor
(CVE-2008-1943 \cite{b1}). Since then, every major hypervisor whether
open-source or commercial, has had something in common: a buffer
overflow or unitialized memory bug in an emulated device allowing for
exploitation by a malicious guest operating system. This raises two
concerns:

\begin{itemize}
\item the attractiveness of attacking emulated devices in a hypervisor,
\item the accessibility of modern techniques to exploit these memory
  bugs when found.
\end{itemize}

\vspace{3mm}
\subsection{Emulated Devices as Targets in Hypervisors}
Like with real computers, devices form the interface between a virtual
machine and the outside world. While hardware-assisted virtualization
allow CPU-intensive tasks to achieve bare-metal speeds, at some point
the guest will need to perform I/O whether sending a network packet or
writing a block to persistent storage. Given I/O becomes a noticeable
performance bottleneck in virtualized systems, hypervisor authors
often optimize in multiple ways:

\begin{itemize}
\item emulate a device in the host kernel to reduce overhead
\item emulate all devices in the same process via threads
\item pass-through access to a real, physical device
\end{itemize}

As a result, emulated devices often require or evolve to acquire
elevated permissions and capabilities, making them a high-value target
for an attacker.

The high-valued nature isn't the only problem: emulated devices must
work with guest device \emph{drivers}. Hypervisor authors must create
virtual hardware from software and, while there exist
virtualization-specific specifications like VirtIO \cite{b2}, the
nature of I/O requires moving guest-supplied data back and forth. A
memory bug easily becomes ``remotely'' triggerable by a guest device
driver or a even a packet coming in to the host from outside.

Given the above, it should come as little surprise that almost all
published virtual machine ``escapes'' \cite{b3} have been as the
result of exploiting emulated devices!

\vspace{3mm}
\subsection{Return-Oriented Programming}
At the same time researchers began to find the first guest-to-host
exploits in hypervisors, other researchers found novel ways to go
beyond simple code-injection techniques and use a program's code
against itself.

Return-Oriented Programming (ROP) \cite{b3} provides a manner to
defeat W$\oplus$X mitigations, commonplace in operating systems with
the support of a hardware ``no-execute'' (NX) page protection bit. ROP
attacks use either the program itself or runtime libraries like
\emph{libc} to execute arbitrary code assembled by finding and
leveraging specificly useful machine instructions called ``gadgets.''
These gadgets, when executed in a particular order (called a
``chain''), allow an attacker to achieve arbitrary code execution.

Multiple approaches exist to help prevent a successful ROP attack,
including Adress Space Randomization (ASLR) and Control Flow Integrity
(CFI) \cite{b4}.

\vspace{3mm}
\subsection{Blind Return-Oriented Programming}
In 2014, \emph{Bittau et. al.} showcased an evolution of ROP, called
``Blind Return-Oriented Programming'' (BROP) \cite{b5}, designed to
overcome ASLR techniques and remotely exploit stack buffer overflows
through information leakage. The authors' techniques of ``stack
reading'' to defeat ASLR and remotely generate ROP chains (sequences
of gadgets) to achieve arbitrary code execution emphasized the
severity of stack or heap overflows in programs: \emph{if it's
remotely accessible and doesn't re-randomize itself on a restart, any
remotely triggerable stack buffer overflow provides a BROP attack
vector.}

How do these concerns apply to hypervisors? While there are no
currently known successful BROP-based attacks on hypervisors, the
concept of \emph{information leakage} still applies. If an attacker
can use a vulnerable emulated device to leak the hypervisor code into
guest memory (assuming no program crash), it's possible to perform a
ROP analysis.

\vspace{5mm}
\section{Defensive Concerns of the Ideal Hypervisor} \label{traits}
Considering the state-of-the-art of ROP-based attacks, what would the
``ideal'' hypervisor consider? BROP has shown it's important to
consider the following:

\vspace{2mm}
\begin{enumerate}
\item Information leakage allows for defeating ASLR.
\item ROP gadgets allow assembling any needed system call.
\item System calls allow lateral movement to take over a host.
\end{enumerate}
\vspace{2mm}

The ideal hypervisor would maximize the complexity of defeating
ASLR. While the trivial approach would be to simply not respawn after
a crash (which isn't a common hypervisor trait anways), one can
propose:

\begin{trait} \label{trait1}
  An information leak in one vulnerable component of a hypervisor must
  not inform on other components of the hypervisor.
\end{trait}

And what about ROP gadgets? They're impossible to completely remove
from the x86 architecture, so while they can be minimized by changes
to compilers, in the ideal case our hypervisor would minimize their
value:

\begin{trait} \label{trait2}
  Compromising a component of a hypervisor must not allow for
  compromising other components of the hypervisor, i.e. a vulnerable
  network device must not allow for compromising other devices.
\end{trait}

Lastly, keeping with the principle of least privilege:

\begin{trait} \label{trait3}
  Escaping a guest, via any means, must force the attacker to then
  exploit the hypervisor itself to gain control of the host.
\end{trait}

While we can harden emulated devices, the difficulty posed to the
attacker should \emph{stay constant or increase} even if they manage
to take control of an emulated device.

\vspace{5mm}
\section{OpenBSD's Hypervisor}
Available since OpenBSD 5.9, released on 29 March, 2016, OpenBSD's
hypervisor consists of three parts:

\vspace{2mm}
\begin{itemize}
\item \texttt{vmm(4)} - the in-kernel virtual machine monitor
\item \texttt{vmd(8)} - the userland virtual machine daemon
\item \texttt{vmctl(8)} - a utility for interacting with vmd
\end{itemize}
\vspace{2mm}

This paper focuses primarily on \texttt{vmd(8)} because it provides
the emulated devices of interest (e.g. VirtIO network devices).

\vspace{3mm}
\subsection{Priviledge Separation in \texttt{vmd}}
While not in the original release of \texttt{vmd}, Reyk Floeter
committed \cite{b6} a redesign to implement a \emph{fork+exec} model
like the one already used in other OpenBSD daemons such as
\texttt{httpd(8)}.

\vspace{2mm}
\begin{figure}
  \includegraphics[width=0.9\linewidth]{privsep-orig.png}
  \caption{vmd(8)'s first priv-sep design.}
\end{figure}
\vspace{2mm}

Shortly thereafter, Reyk added \cite{b7} an additional ``priv''
process intended to run as root and facilitate privledged operations
such as naming host-side \texttt{tap(4)} interfaces to match the name
of the vm.

\vspace{2mm}
\begin{figure}
  \includegraphics[width=0.9\linewidth]{privsep-72.png}
  \caption{vmd(8)'s priv-sep design as of OpenBSD 7.2.}
\end{figure}
\vspace{2mm}


\vspace{3mm}
\subsection{Priviledge Dropping in \texttt{vmd}}
In addition to separating \texttt{vmd} into multiple processes with
dedicated responsibilities, it additionally uses methods to reduce the
privilges of each process. As of OpenBSD 7.2, \texttt{vmd}
incorporates three (3) primary mechanisms for dropping privilege:

\vspace{2mm}
\begin{enumerate}
\item \texttt{setresuid(2)/setresgid(2)} - for changing uid/gid
\item \texttt{chroot(2)} - for isolating file system access
\item \texttt{pledge(2)} - for removing system call access
\end{enumerate}
\vspace{2mm}

In short, the syscalls as early as feasible during program start to
adjust privledges to those in the Table~\ref{tab1}.

\vspace{2mm}
\begin{table}
\begin{center}
\caption{\label{tab1}Existing Privilege Dropping in \texttt{vmd(8)}}
\begin{tabular}{| l | l | l | p{3cm} |}
  \hline
  process & uid/gid & chroot & pledge(s) \\ \hline
  parent & root & \$PWD & stdio rpath wpath proc tty recvfd sendfd getpw chown fattr flock \\ \hline
  control & \_vmd & \$PWD & stdio unix recvfd sendfd \\ \hline
  agentx & \_vmd & / & stdio recvfd unix \\ \hline
  priv & root & \$PWD & -- \\ \hline
  vmm & \_vmd & / & stdio vmm sendfd recvfd proc \\ \hline
  vm & \_vmd & \$PWD & stdio vmm recvfd \\
  \hline
\end{tabular}
\end{center}
\end{table}
\vspace{2mm}

\vspace{3mm}
\subsection{Existing Weaknesses in \texttt{vmd}}
Even with the existing PrivSep design and mitigations, the following
remains true for \texttt{vmd} as of OpenBSD 7.2:

\vspace{2mm}
\begin{enumerate}
\item All devices are emulated in the same guest vm process.
\item Guest vm process creation relies only on \texttt{fork(2)},
  meaning address layouts are the same across guest vms.
\end{enumerate}
\vspace{2mm}

Given our proposed ideal traits in section \ref{traits}, there are
identifiable gaps in the current design of \texttt{vmd} with respect
to device emulation. Consequently, \emph{a compromised device in one
machine exposes all guests under \texttt{vmd} to risk}.


\vspace{5mm}
\section{Hardening a \texttt{vmd} Device}
We'll use the ``ideal traits'' to design our hardening
methodology. Implementing each trait requires using multiple
capabilities of OpenBSD. Let's look at them in order.

\vspace{3mm}
\subsection{Maximizing Randomness}
\label{sec:randomness}
The first step in implementing \emph{Trait 1} is solving for the 2nd
weakness outlined above.

To make each guest have their own address space layout, it should be
as simple as performing the \texttt{exec(2)} part of ``fork+exec,''
right? \texttt{vmd} poses some challenges towards implementation
\emph{because of its existing PrivSep design!}

Firstly, the \emph{vmm} process responsible for spawning new vm
processes doesn't utilize the ``exec'' \texttt{pledge(2)}
promise. This means the \texttt{execvp(2)} syscall is prohibited. But
that's an easy change; just add the promise, right?

Unfortunately, because \emph{vmm} utilizes \texttt{chroot(2)}, it
won't have access to the \texttt{vmd(8)} executable in the file
system. Luckily, we can leverage \texttt{unveil(2)} and the known path
to the vmd executable to approximate the same outcome of minimizing
filesystem access.

As a consequence, we need to make the following changes:

\vspace{2mm}
\begin{enumerate}
\item Remove \emph{vmm}'s \texttt{chroot(2)}.
\item \texttt{unveil(2)} /usr/sbin/vmd in executable mode.
\item Add ``exec'' to \emph{vmm}'s promises.
\item Set the child-side of the \texttt{socketpair(2)} file
  descriptors to not close-on-exec.
\item Implement message passing to bootstrap the new \emph{vm} process
  since we can no longer rely on existing global variables.
\end{enumerate}
\vspace{2mm}

The last part (message passing) requires the most effort. For now, the
following approach is used:

\vspace{2mm}
\begin{enumerate}
\item Add a new \texttt{getopt(3)} argument to indicate we're
  launching a new \emph{vm}-based process and pass that argument
  (\texttt{-V}) when exec'ing.
\item Pass the file descriptor integer for the child-side of the
  \texttt{socketpair(2)}.
\item Use synchronous message passing to send the child \emph{vm}
  process its configuration values.
\end{enumerate}
\vspace{2mm}

At this point, the privledge separation diagram looks like:

\begin{figure}
  \includegraphics[width=0.9\linewidth]{privsep-new1.png}
  \caption{vmd(8)'s priv-sep after adding fork+exec for vm's.}
\end{figure}


\vspace{3mm}
\subsection{Minimizing the Impact -- Isolating a Device}
\texttt{vmd} has had its share of security errata and, like most
hypervisors, most have been related to emulating a network device. How
can we isolate devices?

The most obvious approach is to simply make each device its own
process, each with its own address space like we did with the
\emph{vm} process in section~\ref{sec:randomness}. Ultimately, we want
to achieve a design illustrated in figure~\ref{fig:new2}.

For this, we'll take one of the more complicated and higher value
targets of the emulated devices in \texttt{vmd}: the emulated
\texttt{vio(4)} network device. This poses multiple challenges.

\vspace{2mm}
\begin{figure}
  \label{fig:new2}
  \includegraphics[width=0.9\linewidth]{privsep-new2.png}
  \caption{Isolating multiple VirtIO nics for a vm.}
\end{figure}
\vspace{2mm}


\vspace{2mm}
\subsubsection{Sharing Guest Physical Memory}
The first challenge, and perhaps the most critical, is that the device
needs the ability to read and write directly to guest memory. The
guest memory was already allocated and mapped by the vm process, so we
need to share the pages between vm process and device process.

This change primarily occurs in the vm process. When allocating the
memory for the backing guest memory ranges, we can use
\texttt{shm\_mkstemp(3)} to create a temporary shared memory object
to use when calling \texttt{mmap(2)}. Since the mapping is no longer
anonymous, we can make sure that the file descriptor is not set to
close-on-exec. All that's left is to incorporate the file descriptor
into a configuration message to send to the device after
\texttt{execvp(2)}.


\vspace{2mm}
\subsubsection{Bootstrapping the Device Process}
We need to update the \emph{vm} process to do the
\texttt{fork(2)/execvp(2)} dance. That's easy enough as previously we
added the \texttt{-V} flag and can use the same pattern of passing
configuration data after re-exec. The configuration message needs to
contain the value of the shared memory file descriptor and the already
open \texttt{tap(4)} file descriptor.

Like before, we establish a synchronous communication channel and pass
the file descriptor value via the program arguments so the new device
process can start communicating and receiving the configuration data.

However, unlike the \emph{vmm-to-vm} communication channel, we need an
additional asynchronous one to allow for event-based communication.


\vspace{2mm}
\subsubsection{Communicating with the Device}
The vm process has two threads that may need to communicate with the
device: the vcpu thread that will emulate IN/OUT instructions to the
PCI registers and the event loop thread handling asynchronous events.

For the vcpu thread, we need a \emph{synchronous} channel. When
emulating a PCI register read, the vcpu will expect a response
immediately. (Writes are trivial as no response is needed.) Since this
is the vcpu thread, it cannot leverage any async scheduling of
\emph{imsg}'s since it doesn't own the \texttt{event(3)}
\emph{event\_base} lest it corrupt existing events!

For the event threads, we need an \emph{asynchronous} channel. One
message in particular is relayed from the \emph{vmm} process to the
\emph{vm} process: an update to the host-side MAC. (This occurs at
some time post vm-launch.) Other async messages include pausing and
unpausing the device when the vm is being paused or unpaused.

In either case, \emph{imsg} functions are used to guarantee atomic
delivery.

\vspace{2mm}
\subsubsection{Communicating with the Network}
A functioning \texttt{vio(4)} device requires the following for
operation:

\vspace{2mm}
\begin{itemize}
  \item Two virtqueues (\emph{TX} and \emph{RX}).
  \item An open file descriptor to the host's \texttt{tap(4)} device.
\end{itemize}
\vspace{2mm}

The same mechanisms already in place for using an event loop for
reading off the \texttt{tap(4)} can be reused in this case. The only
difference is now when the device needs to assert or deassert the IRQ,
it needs to broadcast a message to the vm process using the
asynchronous channel.


\vspace{3mm}
\subsection{Escaping into a Void -- Reducing the Device Surface Area}
If all else fails, escaping a \emph{device} should leave the attacker
needing to now find additional exploits to elevate privilege. Let's
assume a bug in the \texttt{vio(4)} device allows a guest-to-host
escape. What are the next potential escalation paths an attacker will
want to chain together to get root on the host?

\vspace{2mm}
\subsubsection{Securing the File System}
One trivial target is the filesystem, either to exfiltrate sensitive
data (like private keys), exploit race conditions, or trick another
program to execute something. The device isn't running as root, so we
can't simply \texttt{chroot(2)} to \texttt{/var/empty}, but we can
leverage \texttt{unveil(2)} and achieve a similar result.

\vspace{2mm}
\subsection{Removing System Calls}
We want to prevent lateral movement and that means preventing system
calls. Thankfully, this is trivial with \texttt{pledge(2)} and,
moreover, we can \emph{further reduce privileges}. Using just the
``stdio'' promise, we reduce to just a minimal subset of syscalls we
need for reading and writing our open file descriptors and managing
our events.

This does mean that an attacker escaping the guest and controlling the
device process can \text{read(2)/write(2)} but the possible targets
are limited to the existing file descriptors (communication channels
with the vm and the host \texttt{tap(4)}). However, no new sockets can
be created nor can any files in the file system be opened.

Any privilege escalation will need to exploit a smaller surface area
than exposed by the vm process and need to rely on kernel bugs, most
likely in this limited area.


\vspace{5mm}
\section{Security! But at what cost?}
If Meltdown and Spectre taught the average user anything it's that
security often comes at the price of performance. What impact do the
proposed architectural changes to \texttt{vmd} have on things like
network performance from the point of view of the guest?

\vspace{3mm}
\subsection{A Simple TCP Benchmark}
While this research doesn't aim to perform a full performance
evaluation at this time, one known area of poor network performance is
when using a TCP performance test program and having the guest act as
the client. Anecdotally, this often shows noticeably worse performance
than the reverse (having the host act as the client).

Utilizing a Lenovo X1 Carbon laptop (10th generation model) with the
an Intel i7-1270P CPU, the observations in Table~\ref{perf} were
observed using \texttt{iperf3(1)} from (chosen as it's available on
both OpenBSD and Alpine). Both guests were allocated 8 GiB of memory
and the recorded result is the best average throughput reported across
3 runs of 60 second duration.

\vspace{2mm}
\begin{table}
\begin{center}
\caption{\label{perf}iperf3 performance test}
\begin{tabular}{| l | l | p{1cm} | l |}
  \hline
  Host & Guest & Bitrate (Gbps) & $\Delta$ (\%) \\ \hline
  -current & OpenBSD 7.2 & 0.86 &  \\ \hline
  prototype & OpenBSD 7.2 & 1.40 & 63\% \\ \hline
  -current & Alpine Linux 3.17 & 1.30 &   \\ \hline
  prototype & Alpine Linux 3.17 & 1.14 & -14\%  \\ \hline
\end{tabular}
\end{center}
\end{table}
\vspace{2mm}

\subsection{Interpretation of Results}
Since \texttt{vmd(8)} does not support multi-processor guests, it
isn't too surprising that we can see a performance improvement from
this design for OpenBSD guests. As a consequence of the message-based
approach, some of the network IO can occur without blocking either the
vcpu thread or event handling thread (responsible for the
\texttt{libevent} event handler) in the main vm process. For instance,
in OpenBSD 7.2, writes by a vcpu to an emulated PCI register will
block while the device emulates them and potentially injects an
interrupt, resulting in a syscall via \texttt{ioctl(2)}.

The potential performance regression in Alpine Linux is not
substantial, but does warrant further investigation.


\vspace{5mm}
\section{Future Work and Considerations}
The prototype has rough edges, specifically around robustness of
lifecycle management. Improving child process handling in the event of
program termination as well as preventing possible messaging deadlocks
at launch would improve viability.

While the performance improvement of networking throughput in OpenBSD
guests was a pleasant surprise, the lack of improvement in Alpine
Linux guests shows there's still potential. The single thread design
of the device process is easier to debug, but prevents simultaneous
transmit and receive processing.

In addition, relying on syscalls and trips through the kernel to
communicate adds extra overhead. While unmeasured at this point, if
it's deemed a bottleneck then exploring messaging via a shared page of
memory and newer Intel process features like \texttt{TPAUSE}
instructions might reduce latency.

Lastly, extending the design to VirtIO block devices would be ideal.


\vspace{5mm}
\section{Conclusions}
OpenBSD contains all the necessary tools for implementing an advanced
hypervisor design that improves security without complicating user
experience. While this paper doesn't explore existing approaches from
systems such as \emph{QEMU}, the idea of isolating devices has been
attempted by multiple hypervisors, but has yet to become the
\emph{default} behavior.

The outlined design and approach for \texttt{vmd(8)} presents a viable
way to bring the isolation needed to take another step towards the
``ideal hypervisor'' without the expense of operator or user
complexity. This approach keeps with the spirit and design of
\texttt{pledge(2)} and \texttt{unveil(2)}: it's the developer's
responsibility to study and improve the program, \emph{not the
user's.}


\vspace{5mm}
\section*{Acknowledgment}
The author would like to thank the AsiaBSDCon program committee for
the opportunity to present on this research. Many thanks also to Mike
Larkin (\texttt{mlarkin@}) for his guidance and mentorship while
hacking on \texttt{vmd(8)}. Lastly, the author's spouse deserves
immense credit for supporting the time and obsession required for
attempting this research.


\vspace{5mm}
\begin{thebibliography}{00}
\bibitem{b1} Red Hat Bugzilla, \url{https://bugzilla.redhat.com/show_bug.cgi?id=443078}, accessed December 2023.
\bibitem{b2} Virtual I/O Device (VIRTIO) Version 1.1. Edited by Michael S. Tsirkin and Cornelia Huck, 11 April 2019, OASIS Committee Specification 01, \url{https://docs.oasis-open.org/virtio/virtio/v1.1/cs01/virtio-v1.1-cs01.html}.
\bibitem{b3} Wikipedia, ``Virtual machine escape'', \url{https://en.wikipedia.org/wiki/Virtual_machine_escape#Previous_known_vulnerabilities}, retrieved 26 December, 2022.
  \bibitem{b4} V. Pappas, ``Defending against Return-Oriented Programming'', Columbia University, 2015.
\bibitem{b5} H. Shacham, ``The Geometry of Innocent Flesh on the Bone:
  Return-into-libc without Function Calls (on the x86)'', Proceedings of the CCS 2007, ACM Press, pp. 552-61, 2007.
\bibitem{b6} OpenBSD GitHub mirror, \url{https://github.com/openbsd/src/commit/bcc679a146056243a2fd52a28182621f893fed4b}
\bibitem{b7} OpenBSD GitHub mirror, \url{https://github.com/openbsd/src/commit/5921535c0be28fd3cf226c9c6a0aa8bb71699acb}
\end{thebibliography}

\end{document}
